\documentclass{article}
\usepackage[utf8]{inputenc}
\usepackage{graphicx}
\usepackage{tikz}
\usepackage{float}
\usepackage{listings}

%Tot això hauria d'anar en un pkg, però no sé com és fa
\newcommand*{\assignatura}[1]{\gdef\1assignatura{#1}}
\newcommand*{\grup}[1]{\gdef\3grup{#1}}
\newcommand*{\professorat}[1]{\gdef\4professorat{#1}}
\renewcommand{\title}[1]{\gdef\5title{#1}}
\renewcommand{\author}[1]{\gdef\6author{#1}}
\renewcommand{\date}[1]{\gdef\7date{#1}}
\renewcommand{\maketitle}{ %fa el maketitle de nou
    \begin{titlepage}
        \raggedright{UNIVERSITAT DE LLEIDA \\
            Escola Politècnica Superior \\
            Grau en Enginyeria Informàtica\\
            \1assignatura\\}
            \vspace{5cm}
            \centering\huge{\5title \\}
            \vspace{3cm}
            \large{\6author} \\
            \normalsize{\3grup}
            \vfill
            Professorat : \4professorat \\
            Data : \7date
\end{titlepage}}
%Emplenar a partir d'aquí per a fer el títol : no se com es fa el package
%S'han de renombrar totes, inclús date, si un camp es deixa en blanc no apareix

\tikzset{
	%Style of nodes. Si poses aquí un estil es pot reutilitzar més facilment
	pag/.style = {circle, draw=black,
                           minimum width=0.75cm, font=\ttfamily,
                           text centered}
}
\renewcommand{\figurename}{Figura}
\title{Pràctica 1}
\author{Ian Palacín Aliana}
\date{23 d'Abril de 2019}
\assignatura{Xarxes}
\professorat{ENRIQUE GUITART BARAUT, CARLOS MATEU PIÑOL}
\grup{}

%Comença el document
\begin{document}
\nocite{*}
\maketitle
\thispagestyle{empty}

\newpage
\pagenumbering{roman}
\tableofcontents
\newpage
\pagenumbering{arabic}

\section{Diagrames d'estructura}
\subsection{Estructura del client}
\subsection{Estructura del servidor}
\section{Manteniment de comunicació}
\subsection{Manteniment de comunicació del client}

Una vegada el client s'ha registrat, passa a la fase de manteniment
de comunicació amb el servidor. Per entendre l'estratègia emprada
primer s'ha d'introduir unes variables importants. La primera
és \textbf{intent}, una variable de tipus enter que s'inicialitzarà a
0 i anirà canviant de valor en funció dels paquets que li arribin (o li
deixin d'arribar). \textbf{Intent} ajudarà a sortir en el moment adient
del bucle principal d'enviament d'ALIVE\textunderscore INF com, per 
exemple, quan es rep un ALIVE\textunderscore REJ (es veuran tots els 
demés casos més endavant).













\subsection{Manteniment de comunicació del servidor}
\section{Diagrama d'estats UDP}
\section{Consideracions}

\end{document}
